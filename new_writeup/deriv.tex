\documentclass{amsart}
\usepackage{amsmath,amsfonts,amssymb,amscd,amsthm,pb-diagram,stmaryrd,mathrsfs}
\usepackage[osf]{baskervillef}
\def\ZZ{\mathbb{Z}}
\def\NN{\mathbb{N}}
\def\RR{\mathbb{R}}
\def\EE{\mathbb{E}}
\def\CC{\mathbb{C}}
\def\TT{\mathbb{T}}
\def\XX{\mathscr{X}}
\def\e{\mathsf{e}}
\def\h{\mathsf{h}}
\def\f{\mathsf{f}}
\def\sF{\mathscr{F}}
\def\sT{\mathscr{T}}
\def\sM{\mathscr{M}}
\def\sR{\mathscr{R}}
\def\fu{\mathfrak{u}}
\def\inv{\tau} % THE involution
\DeclareMathOperator{\Isom}{\mathrm{Isom}}
\DeclareMathOperator{\trans}{\mathfrak{trans}}
\DeclareMathOperator{\pr}{\mathrm{pr}}
\DeclareMathOperator{\id}{\mathrm{id}}
\def\O{\mathsf{O}}
\def\SO{\mathsf{SO}}
\DeclareMathOperator{\tail}{\mathsf{tl}}
\DeclareMathOperator{\head}{\mathsf{hd}}
\DeclareMathOperator{\hor}{\mathsf{hor}}
\DeclareMathOperator{\Hor}{\mathsf{Hor}}
\DeclareMathOperator{\red}{\mathsf{red}}
\def\kin{\mathrm{kin}}
\def\euc{\mathrm{euc}}
\newtheorem{lem}{Lemma}
\newtheorem{prop}{Proposition}
\theoremstyle{definition}
\newtheorem{defn}{Definition}
\begin{document}

\section{Recursion relation}
\subsection{}
Head frame $e$, neck frame $e'$. Head drag $ev$, neck drag $e'v'$:
$$ e'v'= e S_\phi v,\quad e^{-1}e' = S_\phi R_\phi \implies v' = R_{\phi}^{-1}v. $$
Drag-to-motion map $\xi$ (head in frame $e$, neck in frame $e'$):
$$ \xi(v) = f_v(\phi)\partial_\phi + f_v(\phi)\sigma\partial_\phi + \sigma \xi(v') $$
$$ \xi(v) = f_v(\phi)\partial_\phi + f_v(\phi)\sigma\partial_\phi + (1\otimes\sigma\xi)(R_\phi^{-1}\otimes v) $$
where
$$ f_v(\phi) = v_x\sin\phi - v_y\cos\phi.$$
\subsection{}
Complexification and variable change:
$$ z = e^{i\phi},\quad \partial_\phi = iz\partial_z,\quad v^\pm = \epsilon_x \pm i\epsilon_y $$
$$ f_{v^\pm} = \mp i z^{\pm 1},\quad
R_\phi^{-1} \otimes v^{\pm} = z^{\pm 1} \otimes v^{\pm}.$$
Motion basis:
$$ \zeta^+ = \xi(v^+) = z^2\partial_z + z \sigma (z\partial_z) + (1\otimes \sigma\xi)(z\otimes v^+)
= z^2\partial_z + z\sigma(z\partial_z + \zeta^+) $$
$$ \zeta^- = \xi(v^-) =- \partial_z + z^{-1}\sigma(-z\partial z) + (1\otimes\sigma\xi)(z^{-1}\otimes v^-)
= -\partial_z + z^{-1}\sigma(-z\partial_z + \zeta^-). $$
\subsection{}
With $\mathfrak{sl}_2$ notation ($\h$ being half the usual one):
$$ \e = z^2\partial_z,\quad \h = z\partial z,\quad \f = \partial_z $$
$$\zeta^+ = \e + z\sigma(\h + \zeta^+),\quad \zeta^- = -\f + z^{-1}\sigma(-\h + \zeta^-)$$
Involution:
$$ \sigma^\inv=\sigma,\qquad z^\inv = z^{-1},\quad \e^\inv=-\f,\quad \h^\inv=-\h,\quad \f^\inv=-\e,
\qquad
\left(\zeta^{\pm}\right)^\inv=\zeta^\mp $$
 
\section{Configuration space}
\subsection{}
$M$ is the configuration space of infinite snakes in $\RR^2$,
with head at $0$. $\sM$ is the configuration space of infinite snakes in $\EE^2$.
$\sF = \Isom(\RR^2,\EE^2)$ is the orthonormal frame bundle, with each $e\in\sF$ viewed as an isometry 
$\RR^2\to\EE^2$. The structure map sending $e$ to $e(0)$ turns
$\sF\to\EE^2$ into a right $\O(2)$-principal bundle. Furthermore $\sF$
admits a free and transitive left action of $\Isom(\EE^2,\EE^2)$. There is a natural identification
$$ \sM = \sF \times^{\O(2)} M,  $$
and we let $\pi:\sM \to \EE^2$ be the structure map.

\subsection{}
There is an obvious $\Isom(\EE^2,\EE^2)$-equivariant \emph{tail map} $$\widetilde{\tail} : \sM \to \sM,$$ sending
an infinite snake to what remains after cutting off its head segment.
There is also an obvious $\O(2)$-equivariant \emph{head map} $$\head : M \to \TT,$$
sending a snake with head at $0\in\RR^2$ to the other endpoint of its head segment (an element
of the unit circle $\TT \subset \RR^2$). 

\subsection{}
Let $p_i : M \to \RR^2$ send a snake configuration to the position of its $i$-th point,
with $p_0 = 0$ being the snake's nose. We have the induced maps
$$ \widetilde p_i : \sM = \sF\times^{\O(2)} M \to \sF\times^{\O(2)} \RR^2 \simeq \EE^2 \times \EE^2.
$$
In particular, $\widetilde p_0 = \Delta_{\EE^2}\circ\pi$.

\subsection{}
Let $\epsilon_1,\epsilon_2 \in \RR^2$
be the standard oriented orthonormal basis.
For each $u \in \TT$, let $R_u, S_u \in \O(2)$
denote, respectively, the rotation sending $\epsilon_1$ to $u$, and the reflection abouti
the line spanned by $u$. 
Let $$\red : \O(2) \to \O(2)$$ be the
unique group endomorphism of $\O(2)$ with image 
generated by $S_{\epsilon_1}$. Explicitly,
$\red g$ is $\id$ if $\det g = 1$ and $S_{\epsilon_1}$ if $\det g = -1$.
Given an action $\rho$ of $\O(2)$ on 
some object, the corresponding \emph{reduced} action
is the composite $\rho\circ\red$.

\subsection{}
Consider the map
$$\alpha:\sF \times M \to \sF,\quad (e,m) \mapsto e S_{\head m}R_{\head m} + e(\head m).$$
Note that $\alpha$ is $\O(2)$-equivariant,
with the antidiagonal action on the left hand side, and
the reduced right action on the right hand side.  
It is also $\Isom(\EE^2,\EE^2)$-equivariant (via left action on $\sF$).
Since $\O(2)$ acts freely on $M$, there is a unique lift of $\alpha$
to a map $$ \widetilde\alpha : \sF \times M \to \sF \times M $$
descending to $\widetilde\tail : \sM \to \sM$.
Furthermore, since $\widetilde \alpha$ inherits equivariance under
$\Isom(\EE^2,\EE^2)$ from $\alpha$, and since the latter group acts transitively on $\sF$ (and trivially on $M$),
it follows that $\widetilde \alpha$ descends to a map
$$\tail:M \to M$$
such that $\pr_M \circ \widetilde\alpha = \tail\circ\pr_M$. 
Furthermore, $\tail$ is $\O(2)$-equivariant with the
standard action on the left hand side, and the reduced action on the right hand side.

\subsection{}
The map
$$ \langle \head,\tail\rangle : M \to \TT \times M $$
is an isomorphism, expressing $M$ coinductively as an infinite-dimensional torus (of course there are as many such presentations, as cofiltration preserving automorphisms of $\TT^\infty$).
We consider the associated Lie algebra homomorphisms
$$
\iota_{\head} : \XX(\TT) \to \XX(M),\quad \iota_{\tail} : \XX(M) \to \XX(M)
$$
such that the image of $\iota_{\head}$ annihilates that of $\tail^* :C^\infty(M) \to C^\infty(M)$,
and the image of $\iota_{\tail}$ annihilates that of $\head^* : C^\infty(\TT) \to C^\infty(M)$.

\section{Vector fields}
\subsection{}
The non-holonomic constraints induce a connection $\nabla^\sM_\kin$ in the fibre bundle
$\sM \to \EE^2$. On the other hand, the flat Euclidean structure of $\EE^2$ induces
a principal connection $\nabla^\sF_\euc$ in $\sF \to \EE^2$. That in turn descends
to another connection $\nabla^\sM_\euc$ in $\sM \to \EE^2$. 
Given a vector $v \in T_p\EE^2$, the difference between the horizontal lifts
defines a vector field on the fibre:
$$ \hor^\sM_\kin v - \hor^\sM_\euc v \in \XX(\sM_p). $$
In terms of associated bundles, this becomes a map:
$$ \widetilde\xi: T\EE^2 \simeq \sF\times^{\O(2)}\RR^2 \to \sF \times^{\O(2)} \XX(M) $$
By construction, $\widetilde\xi$ is invariant under the left action
of $\Isom(\EE^2,\EE^2)$, whence it arises from an $\O(2)$-equivariant map
$$ \xi : \RR^2 \to \XX(M). $$

\subsection{}
Given a pair $(p,p') \in \EE^2\times\EE^2$,
let $\widetilde S_{(p,p')}:T_p\EE^2\to T_{p'}\EE^2$
be the linear map sending a vector $v \in T_p\EE^2$
to its reflection about the line spanned by $(p,p')$,
translated to $p'$.
The connection $\nabla^\sM_\kin$ is uniquely
determined by the following properties:
\begin{enumerate}
        \item $\Hor^\sM_\kin \subset T\sM$  is preserved by 
                $T\widetilde\tail$,
        \item $(T\pi \circ T_m\widetilde\tail) (\hor^\sM_\kin v) = \widetilde S_{\widetilde p_1(m)} v$ for $v\in T_p\EE^2$, $m \in \sM_p$.
\end{enumerate}


\subsection{}
Let $\omega \in\Lambda^2\RR^{2*}$
denote the unit positive volume form.
Consider the vector field 
$$ r \in \XX(\TT),\quad r_u = \omega(u)^\sharp $$
and the map
$$ f : \RR^2 \to C^\infty(\TT),\quad f_v(u)= \omega(v,u).$$
Explicitly, $\omega=\epsilon_1^*\wedge\epsilon_2^*$,
$r(u) = u_1 \epsilon_2 - u_2\epsilon_1$,
and $f_v(u) = v_1u_2-v_2u_1$.  Consider also the usual map
$$ \sR : \TT \to \SO(2),\quad \sR(u) = R_u  $$
identifying the torus with the circle group.

\subsection{Claim}
$$
\xi(v) = \iota_{\head} (f_v r) + (\head^* f_v) \iota_{\tail} (\iota_{\head} r)
+ \underbrace{(\head^* \otimes \iota_{\tail}\xi) (\sR^{-1}\otimes v)}_{
\sum v^j (\head^*{\sR_j}^i) \iota_{\tail} \xi(\epsilon_i)}.
$$



\end{document}
