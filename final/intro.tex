\section{Introduction}
\subsection*{}
The `infinite snake' is a non-holonomic planar mechanical system:
an infinite chain of rigid segments of unit length, with kinematics
constrained by the requirement that the instantaneous velocity of
a segment's midpoint be parallel to the segment itself.

Geometrically, these constraints define a non-integrable vector distribution $D \subset T\sM$ (a
sub-bundle of the tangent bundle) on the system's configuration space $\sM$ (an infinite-dimensional
manifold). It has rank 2, and is spanned by a distinguished pair of vector fields $X,Y \in \XX(\M)$. 
Their iterated Lie brackets satisfy non-trivial relations, and the main result of this paper is
an \emph{explicit isomorphism} between the Lie subalgebra generated by $X,Y$ in $\XX(\M)$, and
the positively graded ideal of a \emph{twisted loop algebra} of
$\so(2,1)$.
Furthermore, this Lie algebra can also be identified with the \emph{symbol} of $D$ at a generic point,
a fundamental invariant of a non-integrable distribution.

In the remainder of this introductory section, I will sketch the broader context of this work,
state the main results in a precise form, and outline the contents of the article.

\subsection{}
The infinite snake has its origins in a `geometric robots' project proposed by Pawel Nurowski around 2014.
I will briefly explain its two fundamental ideas:
\begin{enumerate}
        \item certain non-integrable distributions can be viewed as sources of rich geometry, 
        \item a large supply of interesting non-integrable distributions is offered
              by simple, toy-like mechanical systems.
\end{enumerate}

Traditionally, non-integrable distributions arise in areas such as \textsc{PDE}s, non-holonomic mechanics,
and control theory. Pursuing their intrinsic properties, one quickly arrives at the problem of local 
equivalence (including auto-equivalence, i.e. symmetry).
That in turn forces one to consider local invariants of a non-integrable distribution, in particular 
its \emph{symbol}:
a bundle of graded Lie algebras, associated with an increasing 
filtration of the tangent bundle induced by the
distribution (possibly after removing a singular subset). 

The filtration $F^\bullet TM$ induced by $D \subset TM$
satisfies $F^0TM=0$, $F^1TM=D$ and is minimal with the property that $[\Gamma(F^iTM), \Gamma(F^jTM)]
\subset \Gamma(F^{i+j}TM)$. There is a smallest $k>0$ such that $F^kTM$ is involutive, and 
we shall assume that actually $F^k TM=TM$ (i.e. the filtration is exhaustive); otherwise,
we would restrict to its leaves. The sequence $(\rk F^i TM)_{1 \le i \le k}$ is called the
\emph{growth vector} of $D$.

Now, the symbol is the associated graded object of $F^\bullet TM$, a graded vector bundle over $M$.
Its fibres have a natural structure of graded nilpotent
Lie algebras, generated in degree one, 
with graded components in degrees $1,\dots,k$ of dimensions encoded by the increments
of the growth vector (padded by zero on the left). 
We may visualise this data as a map from $M$ into the moduli space of such 
Lie algebras. Distributions for which this map is constant are of particular interest: by the
work of Tanaka and others, they allow one to construct a Cartan geometry
over the base manifold. 
Having fixed the isomorphism class of symbol algebras, this construction
can be made functorial, thus reducing the equivalence problem for such
distributions to the equivalence problem for certain
Cartan geometries.

From this point of view geometry is a means to the end of classifying non-integrable distributions,
with further practical applications in mind. The first idea underlying the `geometric robots' project
is to take the opposite view: non-integrable distributions, through the process described above, can be
used to generate geometric structures of much richer nature. For example,
generic rank two distributions in dimension five induce on their underlying fivefold a Cartan geometry
modeled on a homogeneous space for the split real form of the exceptional group $\mathsf{G}_2$, along
with a wealth of associated objects, including a conformal class of signature $(3,2)$.

How then does one produce interesting distributions? Nurowski was inspired by the work of Masato Ishikawa,
exploring certain robotic systems from an engineering perspective, yet revealing remarkably deep geometric
properties. In particular, Ishikawa studied (also experimentally!) planar robots built from rigid elements
(segments or polygons), connected at adjacent vertices by active rotating joints, and possibly equipped
with passive wheels. The condition that the wheels roll without skipping or skidding introduces non-holonomic
constraints, cutting out a non-integrable distribution in the tangent bundle of the robot's configuration space.
One such robot, the \emph{three-segment snake}, gives rise to a generic rank two distribution in dimension five,
and thus to a $\mathsf{G}_2$ Cartan geometry. Another, the \emph{trident snake}, induces a generic rank three
distribution in dimension six, corresponding to an $\mathsf{SO}(4,3)$ Cartan geometry. 
One may easily imagine other `snakes', i.e. planar robots
composed of rigid segments and passive wheels with a no-skidding-or-skipping constraint. These systems have
a combinatorial flavour, but also include continuous parameters, 
namely segment lenghts and wheel positions, giving rise to
families of geometries (perhaps with special, exceptionally symmetric, members). That is the second idea underlying the `geometric robots' project.

\subsection{}\label{subsec:table}
How much can be deduced from a `snake's combinatorial structure? Certainly the dimension of its configuration
space, and the rank of the constraint distribution -- the initial and final entries of the growth vector (assuming the filtration of the tangent bundle is exhaustive). What about the entire growth vector? I was approaching
this question in 2014, attempting to understand what happens as a `snake' is built inductively, by appending
new segments one by one. Arguably the simplest construction of this sort is the one that resembles
an actual snake: a sequence of segments, each attached to the end of the previous one. With each segment
equipped with a wheel in its very middle, these systems are what we'll call $n$-segment snakes (including
the $3$-segment snake mentioned above in relation to $\mathsf{G}_2$).

Here are the growth vector increments (i.e. ranks of graded
components of the symbol) for the first few:
\begin{center}\begin{tabular}{@{}lclllllll@{}}
        $n$ & & \multicolumn{7}{c}{growth vector increments} \\
        \midrule
        $1$ & & $2$ & $1$ \\
        $2$ & & $2$ & $1$ & $1$ \\
        $3$ & & $2$ & $1$ & $2$ \\
        $4$ & & $2$ & $1$ & $2$ & $1$ \\
        $5$ & & $2$ & $1$ & $2$ & $1$ & $1$ \\
        $6$ & & $2$ & $1$ & $2$ & $1$ & $2$ \\
        $7$ & & $2$ & $1$ & $2$ & $1$ & $2$ & $1$ \\
        $8$ & & $2$ & $1$ & $2$ & $1$ & $2$ & $1$ & $1$\\
        $9$ & & $2$ & $1$ & $2$ & $1$ & $2$ & $1$ & $2$ \\
\end{tabular}
\end{center}
The pattern is evident: the growth vector increments are an alternating
sequence of $2$ and $1$, except perhaps for the terminal entry, where
a $1$ can appear instead of $2$. The initial entry is $2$, and that's the rank
of the constraint distribution. The sum of the entries is $n+2$, and that's the
dimension of the configuration space. Explaining this pattern had been the main
motivation behind this paper.

\subsection{}
Let us quickly sketch the connection between Theorem~\ref{thm:main} and the above table
of growth vector increments. We need to consider $n$-segment snakes for $n \in \NN \cup \{\infty\}$.
The $n$-segment snake is described by a configuration space $\sM_n$ together with a distribution
$D_n \subset T\sM_n$. In particular, $\sM_0 \simeq \RR^2$ and $D_0 = T\sM_0$.
We may view each finite snake as a subsystem of the infinite snake,
consisting of the snake's `head', along with a finite number of subsequent segments. Shorter finite
snakes are in turn subsystems of longer finite snakes, so that eventually we have projection maps
\[ \pi^n_{n'} : \sM_n \to \sM_{n'},\quad 0 \le n' \le n \le \infty \]
such that $T\pi^n_{n'} : T\sM_n \to T\sM_{n'}$ maps $D_n$ into $D_{n'}$. 
It turns out that in fact the latter map induces isomorphisms between the fibres of $D_n$ and $D_{n'}$.
Ultimately, taking $n'=0$ we find that $\rk D_n = 2$ for all $n \in \NN \cup \{\infty\}$.

The translation group $\RR^2$ acts on all the $\sM_n$, so that the projections $\pi^n_{n'}$
are equivariant, and the distributions $D_n$ are invariant. In particular, the action trivialises
$T\sM_0 \simeq \sM_0 \times \RR^2$; combining this with isomorphisms $D_n \simeq \pi^{n*}_{n'} D_{n'}$
induced by the projections, we obtain trivialisations $D_n \simeq \sM_n \times \RR^2$. Eqiuvalently, for each 
$n \in \NN \cup \{\infty\}$
we get a map
\[ \RR^2 \to \XX(\sM_n) \]
whose image spans $D_n$ at each point. Letting $\FL(\RR^2)$ denote the free Lie algebra on the vector
space $\RR^2$ (i.e. the free Lie algebra on two generators), the above map extends to a Lie algebra homomorphism
\[ \phi_n : \FL(\RR^2) \to \XX(\sM_n) \]
by the universal property of $\FL$.

The algebra $\FL(\RR^2)$ is naturally graded; we may also view it as filtered, with $F^\ell \FL(\RR^2)$
spanned by elements of degree $\le \ell$. On the other hand, letting $m \in \sM_n$
be a regular point with an open neighbourhood $U \subset \sM_n$
over which $D_n$ induces a well-defined filtration $F^\bullet TU$ by vector sub-bundles,
we may set $F^\ell\XX(U) = \Gamma(F^\ell TU)$. Then the map
\[ \res_U \circ \phi_n : \FL(\RR^2) \to \XX(U) \]
is compatible with the filtrations, and the image of $F^\ell\FL(\RR^2)$
spans $F^\ell TU$ at each point. Thus, via evaluation at $m$, we have a surjective  homomorphism
of \emph{graded} Lie algebras
\[ \phi_{n,m}^{\Symb} : \FL(\RR^2) \to \Symb_m D_n \]
onto the \emph{symbol} of $D_n$ at $m$. 
To understand the latter, we need to control $\ker\phi_{n,m}^{\Symb}$.
So far, we have:
\[ \ker\phi_n \subset \ker \phi_{n,m}^{\Symb}.  \]

In fact, the image of $\phi_n$ is contained in the Lie subalgebra $\XX^{\pr}(\sM_n)$
of \emph{projectable} vector fields
with respect to all the projections $\pi^n_{n'}$, $n'\le n$: these are the vector fields on $\sM_n$
whose action as derivations preserves each subalgebra $\pi^{n*}_{n'}C^\infty(\sM_{n'}) \subset C^\infty(\sM_n)$.
Projection maps $\pi^n_{n'}$ induce well-defined Lie algebra homomorphisms
\[ \pi^n_{n'*} : \XX^{\pr}(\sM_n) \to \XX^{\pr}(\sM_{n'}), \]
and we have by construction that\footnote{
By the way, if the reader is uncomfortable working with the infinite-dimensional
$\sM_\infty \simeq \varprojlim_{n<\infty} \sM_n$, limit of a diagram of spaces (manifolds?),
it is enough accept the more palatable
$\XX^{\pr}(\sM_\infty) \simeq \varprojlim_{n<\infty} \XX^{\pr}(\sM_n)$,
limit of a diagram of Lie algebras.}
\[ \phi_{n'} = \pi^n_{n'*} \phi_n\,\quad 0\le n'\le n \le \infty. \]
Hence $\ker \phi_n \subset\ker\phi_{n'}$ for $n' \le n$, and
in particular
\[ \ker\phi_\infty \subset \ker\phi_n \subset \ker\phi_{n,m}^{\Symb}. \]

Even further, $\phi_n$ does in fact factor through $\XX^{\pr}(\sM_n)^{\RR^2}$, the
subspace of \emph{translation-invariant} projectable vector fields on $\sM_n$ (of course preserved by the $\pi^n_{n'*}$).
Specialising to $n=\infty$ and omitting projectability, let us consider the quotient map
\[ 0 \to \RR^2 \to \XX(\sM_\infty)^{\RR^2}\xrightarrow{q} \XX(\sM_\infty/\RR^2). \]
Theorem~\ref{thm:main} states that the image of $q \circ \phi_\infty$ is isomorphic to
the graded Lie algebra
\[ \fL_+ = \bigoplus_{i>0} \fL_i,\quad \fL_{2i} \simeq \so(2),\quad \fL_{2i+1} \simeq \RR^2 \]
with Lie brackets induced by the Cartan decomposition 
\[ \so(1,2) = \so(2) \oplus \RR^2. \]
We may restate it as follows.
\begin{thm*}
There is a commutative diagram of Lie algebra homomorphisms
\[\begin{tikzcd}
        \FL(\RR^2) \arrow[d, "\psi", two heads] \arrow[r,"\phi_\infty"] & \XX(\sM_\infty)^{\RR^2} \arrow[d,"q",two heads] \\
        \fL_+ \arrow[r,"\iota",rightarrowtail] & \XX(\sM_\infty/\RR^2) 
\end{tikzcd}\]
where $\psi$ is a graded surjection induced by the identification $\fL_1=\RR^2$ and the universal property of $\FL$,
while $\iota$ is an injection. 
\end{thm*}

In particular, $\ker \phi_\infty \subseteq \ker(q \circ \phi_\infty) = \ker\psi$. In fact, it turns out that
the inclusion is an equality, so that eventually
\[ \ker\psi = \ker\phi_\infty \subset \ker\phi_{n,m}^{\Symb}. \]
Put differently, the symbol of $D_n$ at a general point $m \in \sM_n$
is a \emph{graded quotient} of $\fL_+$. Now the puzzle is essentially reduced to the 
following observation:
\begin{prop*} For each $n \in \NN$, there is, up to isomorphism, precisely one $(n+2)$-dimensional graded quotient 
        Lie algebra of $\fL_+$ generated in degree $1$. The ranks of its graded components follow the
        pattern of the table in Subsection~\ref{subsec:table}.
\end{prop*}

\subsection{}
The proof of Theorem~\ref{thm:main} is, ultimately, computational. The calculation is however greatly
simplified by exploiting \emph{symmetries} of the main system of interest: the infinite snake. We've already
employed \emph{translational} symmtery, working with vector fields on the reduced configuration space, which
we'll now denote simply by $M=\sM_\infty /\RR^2$. We'll also write $\xi = q \circ\phi_\infty$.
Of course, the original system enjoyed full \emph{Euclidean}
symmetry, whence we're still left with the rotation group $\O(2)$ acting on $M$, so that the
homomorphism
\[ \xi : \FL(\RR^2) \to \XX(M) \]
becomes $\O(2)$-equivariant with respect to the natural induced actions on the domain and codomain.

It is natural to diagonalise the $\SO(2)$-actions. This requires
complexifying the above situation:
\[ \xi : \FL(\CC^2) \to \XX_\CC(M) = \CC \otimes \XX(M)\]
where we extend $\xi$ to a complex-linear map. 
Let $v^\pm = e_1 \pm ie_2$ be an isotropic basis of $\CC^2$,
with a suitable generator of $\so(2,\CC)$ acting as $v^\pm\mapsto\pm v^\pm$.
We are interested in the Lie subalgebra 
generated by
\[ \zeta^\pm = \xi(v^\pm) \in \XX_\CC(M). \]
These turn out to satisfy certain remarkable identities.

There's a further semi-symmetry we haven't yet exploited: the tail of an infinite snake
is itself an infinite snake -- and it behaves as such. That is, the resulting map
$\tail:\sM_\infty \to \sM_\infty$ sends $D_\infty \subset T\sM_\infty$ to itself.
Observe that the data corresponding to $\tail$ at the finite $n$ level
is a map $\sM_n \to \sM_{n-1}$; this only becomes a self-map at $n=\infty$.
Of course, $\tail$ is not invertible -- hence the prefix \emph{semi}.\footnote{One could
go further and consider doubly infinite snakes, with $\tail$ becoming an honest symmetry.}

Much of Sections \ref{sec:deriving} and \ref{sec:abstracting} is devoted to 
working out how the tail map interacts with $\xi$, and thus $\zeta^\pm$. While this is
not difficult, and quite entertaining to see intuitively, some care has to be taken to
write down correct equations for $\xi$ and $\zeta^\pm$. 
In short, one considers a \emph{twist} $\ttl:M \to M$ of the tail map descended to $M$,
along with the head map $\hd : M \to \TT$: these define an isomorphism
\[ \langle \hd,\ttl\rangle : M \xrightarrow{\simeq} \TT \times M. \]
Using its inverse, we consider the pair of injective homomorphisms
\[ \XX_\CC(\TT) \xrightarrow{\iota} \XX_\CC(M) \xleftarrow{\sigma} \XX_\CC(M) \]
of Lie algebras. In particular, the `shift map'
$\sigma$ sends an infinitesimal motion of the infinte snake,
now viewed as the tail of another infinite snake, to
an infinitesimal motion of the latter, keeping the head immobile. 
Now, we have the relation:
\begin{equation}\label{eq:rec-rel-first}
        \fbox{\zeta^+ = z^2\partial_z + z\sigma\left(\zeta^+ + z\partial_z \right) }
\end{equation}
where $z \in C^\infty(\TT, \CC)$ is the standard complex coordinate
on the torus, and we've omitted the maps $\hd^* : C^\infty(\TT,\CC) \to C^\infty(M,\CC)$
and $\iota : \XX_\CC(\TT) \to \XX_\CC(M)$ from notation.

Here's a number of observations regarding \eqref{eq:rec-rel-first}.
\begin{enumerate}
        \item 
                The operator $1 - z\sigma : \XX_\CC(M)\to\XX_\CC(M)$ is invertible,
                so that \eqref{eq:rec-rel-first} determines $\zeta^+$ uniquely.
        \item 
                There is a reflection in $\O(2)$ inducing involutions
                on $\XX_\CC(M)$ and related spaces, denoted by $\tau$,
                such that $\tau\sigma=\sigma\tau$, $\tau z = z^{-1}\tau$, and
                $\zeta^- = \tau\zeta^+$. Hence \eqref{eq:rec-rel-first} determines $\zeta^-$ as well.
        \item 
                The vector fields $z^2\partial_z$, $z\partial_z$ and $\tau(z^2\partial_z) = \partial_z$,
                identified with their images under $\iota$,
                form a Lie algebra isomorphic to $\sl(2,\CC)$, a complexification of the
                $\so(1,2)$ giving rise to the loop algebra $\fL_+$ eventually controlling the 
                entire situation. This is a first glimpse of the role of $\so(1,2)$ in the infinite
                snake's kinematics.
\end{enumerate}
In light of the latter observation, one may contemplate three factors in the infinite snake's generalised
symmetries: obvious Euclidean symmetry, `shift' semi-symmetry, and `hidden' $\so(1,2)$. 

One may view \eqref{eq:rec-rel-first}
as constructing an object over the infinite-dimensional space $M$
in terms of finite-dimensional data -- a pair of elements of $\sl(2,\CC)$ embedded in $\XX_\CC(M)$.
That is the `taming' of the snake's infinite aspect by means of the shift semi-symmetry and associated (co)recursion 
(cf. the picture of Ouroboros eating its tail on the title page).

\endinput
